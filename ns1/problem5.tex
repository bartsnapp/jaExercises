\documentclass{ximera}
\author{Jont Allen}
\begin{document}
\begin{problem}
    Find the largest prime number $p_{\text{max}}$ that can be stored on an Intel 64-bit computer. 
    \end{problem}
    
    \begin{problem}
    Recall that the largest unsigned integer representable on a 64-bit system is $2^{64} - 1$. 
    Use a prime-finding algorithm (e.g., in MATLAB or Python) to identify the largest prime within this range.

    The largest prime is $\answer{18446744073709551615}$.
    \begin{feedback}[correct]
        The largest unsigned integer representable on a 64-bit system is $2^{64} - 1$, which equals $18,446,744,073,709,551,615$. 
Using a prime-finding algorithm, the largest prime less than this value is $18,446,744,073,709,551,557$.
    \end{feedback}
    \end{problem}
    
    \begin{problem}
    Explain why $p_{\text{max}}$ is unique and discuss its significance in modern computing, 
    particularly in cryptography and number theory.
    \begin{multipleChoice}
        \choice[correct]{I've thought about this.}
        \choice{I have not thought about this.}
    \end{multipleChoice}
    \begin{feedback}[correct]
$p_{\text{max}}$ is unique because it is the largest prime that fits within the range of a 64-bit unsigned integer. 
In cryptography, large primes are crucial for generating secure keys in encryption algorithms like RSA. 
Their uniqueness ensures deterministic behavior when generating prime-based keys.
\end{feedback}


    \end{problem}
\end{document}