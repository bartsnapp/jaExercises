\documentclass{ximera}
\author{Jont Allen}
\begin{document}
    A special notation called \texttt{floor} is used to round numbers down to the nearest integer. 
    Using the floor function, denoted as $\lfloor x \rfloor$, compute the following:    
    \begin{problem}
    Compute $\lfloor 3.7 \rfloor$.
    \[
        \lfloor 3.7 \rfloor = \answer{3}
    \]
    \begin{feedback}[correct]
    The floor of $3.7$ is $3$, because the largest integer less than or equal to $3.7$ is $3$.
    \end{feedback}
    \end{problem}
    
    \begin{problem}
    Compute $\lfloor -2.5 \rfloor$.
    \[
        \lfloor -2.5 \rfloor = \answer{-3}
    \]
    \begin{feedback}[correct]
    The floor of $-2.5$ is $-3$, because the largest integer less than or equal to $-2.5$ is $-3$.
\end{feedback}

    \end{problem}
    
    \begin{problem}
    Compute $\lfloor 0.9 \rfloor$.
    \[
        \lfloor 0.9 \rfloor= \answer{0}
    \]
    \begin{feedback}[correct]
    The floor of $0.9$ is $0$, because the largest integer less than or equal to $0.9$ is $0$.
\end{feedback}

    \end{problem}
    
    \begin{problem}
    Explain why $\lfloor x \rfloor \leq x$ for all real numbers $x$.
    \begin{multipleChoice}
        \choice[correct]{I've thought about this.}
        \choice{I have not thought about this.}
    \end{multipleChoice}
    \begin{feedback}[correct]
    The floor function, $\lfloor x \rfloor$, is defined as the largest integer less than or equal to $x$. 
    By definition, it cannot exceed $x$ since it is either equal to $x$ (if $x$ is an integer) or strictly less than $x$ (if $x$ is not an integer).
\end{feedback}

    \end{problem}
\end{document}