\documentclass{ximera}
\author{Jont Allen}
\begin{document}
\begin{problem}
    Explain why the function \texttt{isprime} returns \(1\) for prime numbers and \(0\) for non-prime numbers.
    Use examples \(2\), \(3\), and \(4\) to illustrate this behavior.
    \[
    \text{isprime(2): } \answer{1}, \text{isprime(3): } \answer{1}, \text{isprime(4): } \answer{0}
    \]
    \begin{feedback}[correct]
    The function \texttt{isprime(n)} returns \(1\) (true) if \(n\) is prime and \(0\) (false) if \(n\) is not. For example, \(2\) and \(3\) are prime numbers, so the function returns \(1\). For \(4\), which is not prime, the function returns \(0\).
    \end{feedback}
\end{problem}

\begin{problem}
    Discuss the efficiency of \texttt{isprime} for large numbers. How does it compare to other prime-checking algorithms?
    \[
    \text{Provide an efficiency analysis.}
    \]
    \begin{feedback}[correct]
    The \texttt{isprime} function is efficient for moderately large numbers but may become slower for very large inputs. Other algorithms, such as the Miller-Rabin primality test, are faster for extremely large numbers due to their probabilistic nature.
    \end{feedback}
\end{problem}
\end{document}